% resumo em inglês
\setlength{\absparsep}{18pt} % ajusta o espaçamento dos parágrafos do resumo
\begin{resumo}[Abstract]
 \begin{otherlanguage*}{english}
   This paper analyzes the development of applications with microservice architecture, exposing the characteristics of this architectural approach and gathering and discussing practices used in the development of applications that use it, through literature research and review. Some tools will also be discussed and used for the development of an example application. The result is a set of common characteristics, well established practices and useful tools in the development of such applications. Furthermore, it has been identified that certain practices have subjetive circumstances and must be pondered before applied, for they are not always considered favorable, sometimes being judged positive by some authors and negative by others.
   
  %  it has been identified a certain level of disagreement between respected authors in this area about the practice of starting with a simpler architectural approach, such as the monolithic one, before utilizing microservices, issue in which there is more space for discussions and case studies.

   \vspace{\onelineskip}
 
   \noindent 
   \textbf{Keywords}: software architecture. microservices. development. practices. tools.
 \end{otherlanguage*}
\end{resumo}