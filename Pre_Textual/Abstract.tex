% resumo em inglês
\setlength{\absparsep}{18pt} % ajusta o espaçamento dos parágrafos do resumo
\begin{resumo}[Abstract]
 \begin{otherlanguage*}{english}
   Choosing an application architecture involves tradeoffs and decisions that can be complex, and it is essential to understand its benefits, disadvantages, and challenges so that they can be applied to the context in question. Although there is no formal definition for the microservices architecture, this paper highlights its most frequent characteristics, such as componentization and high complexity; commonly applied practices, such as CI/CD and monitoring; and tools frequently used in its development and operation, through research and bibliographic review. A topic with room for further discussion was also identified, on the practice of starting with a monolithic architecture and only later migrating to microservices. Furthermore, an \emph{E-commerce} web application with microservices architecture was developed, using several of the practices and tools presented. The result is a set of common characteristics, well-established practices, and useful tools in the development of such applications, as well as a practical example that demonstrates how these can be applied and how this architecture behaves.
   
  %  it has been identified a certain level of disagreement between respected authors in this area about the practice of starting with a simpler architectural approach, such as the monolithic one, before utilizing microservices, issue in which there is more space for discussions and case studies.

   \vspace{\onelineskip}
 
   \noindent 
   \textbf{Keywords}: software architecture. microservices. development. practices. tools.
 \end{otherlanguage*}
\end{resumo}