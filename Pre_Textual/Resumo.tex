% resumo em português
\setlength{\absparsep}{18pt} % ajusta o espaçamento dos parágrafos do resumo
\begin{resumo}

Este trabalho examina o desenvolvimento de aplicações com arquitetura de microsserviços, discutindo as características comuns desta abordagem arquitetural e reunindo boas práticas e ferramentas usadas no seu desenvolvimento, por meio de pesquisa e revisão bibliográfica. O resultado é um conjunto de boas práticas e ferramentas bem consolidadas no desenvolvimento de tais aplicações. Também foi identificado certo nível de discordância sobre a prática de começar a aplicação por uma abordagem arquitetural mais simples tal como a monolítica antes de adotar os microsserviços, ideia em que há mais espaço para discussões e estudos de caso. 

% Segundo a \citeonline[3.1-3.2]{NBR6028:2003}, o resumo deve ressaltar o objetivo, o método, os resultados e as conclusões do documento. A ordem e a extensão destes itens dependem do tipo de resumo (informativo ou indicativo) e do tratamento que cada item recebe no documento original. O resumo deve ser precedido da referência do documento, com exceção do resumo inserido no próprio documento. (\ldots) As palavras-chave devem figurar logo abaixo do resumo, antecedidas da expressão Palavras-chave:, separadas entre si por ponto e finalizadas também por ponto.

 \textbf{Palavras-chave}: arquitetura de software. microsserviços. boas práticas. ferramentas.
\end{resumo}