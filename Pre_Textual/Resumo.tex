% resumo em português
\setlength{\absparsep}{18pt} % ajusta o espaçamento dos parágrafos do resumo
\begin{resumo}

A escolha da arquitetura de uma aplicação envolve \emph{tradeoffs} e decisões que podem ser complexas, sendo essencial entender seus benefícios, desvantagens e desafios para que possam ser aplicados ao contexto tratado. Embora não haja uma definição formal para a arquitetura de microsserviços, o presente trabalho destaca suas características mais frequentes, como componentização e alta complexidade; práticas comumente aplicadas, como CI/CD e monitoramento; e ferramentas frequentemente usadas no seu desenvolvimento e operação, por meio de pesquisa e revisão bibliográfica. Também foi identificado um tópico com espaço para mais discussões, sobre a prática de começar com uma arquitetura monolítica para só depois migrar para microsserviços. Ademais, foi desenvolvida uma aplicação web de \emph{E-commerce} com arquitetura de microsserviços, utilizando várias das práticas e ferramentas apresentas. O resultado é um conjunto de características comuns, práticas bem consolidadas e ferramentas úteis no desenvolvimento de tais aplicações, assim como um exemplo prático que demonstra como essas podem ser aplicadas e como se comporta essa arquitetura.

% Assim sendo, este trabalho aborda de forma abrangente a arquitetura de microsserviços, porém reconhece a necessidade de adaptação contínua às novas tecnologias e práticas que surgem constantemente na área da computação.

% A escolha da arquitetura de uma aplicação envolve decisões complexas e \emph{tradeoffs}, sendo essencial entender seus benefícios, riscos, desvantagens e desafios para aplicá-los ao contexto tratado. O presente trabalho analisa o desenvolvimento de aplicações com arquitetura de microsserviços, expondo as características desta abordagem arquitetural e apresentando e contextualizando práticas usadas no desenvolvimento de aplicações que a usa, por meio de pesquisa e revisão bibliográfica. Também são apresentadas e usadas algumas ferramentas para o desenvolvimento de uma aplicação exemplar com arquitetura de microsserviços. O resultado é um conjunto de características comuns, práticas bem consolidadas e ferramentas úteis no desenvolvimento de tais aplicações, assim como um exemplo prático. Ademais, foi identificado que certas práticas têm circunstâncias subjetivas e devem ser ponderadas antes de aplicadas, pois nem sempre são consideradas favoráveis, por vezes sendo julgadas positivas por alguns autores e negativas por outros.

% Ademais, foi identificado certo nível de discordância entre autores respeitados nesta área sobre a prática de começar a aplicação por uma abordagem arquitetural mais simples, tal como a monolítica, antes de adotar os microsserviços, questão em que há mais espaço para discussões e estudos de caso. 

% Segundo a \citeonline[3.1-3.2]{NBR6028:2003}, o resumo deve ressaltar o objetivo, o método, os resultados e as conclusões do documento. A ordem e a extensão destes itens dependem do tipo de resumo (informativo ou indicativo) e do tratamento que cada item recebe no documento original. O resumo deve ser precedido da referência do documento, com exceção do resumo inserido no próprio documento. (\ldots) As palavras-chave devem figurar logo abaixo do resumo, antecedidas da expressão Palavras-chave:, separadas entre si por ponto e finalizadas também por ponto.

 \textbf{Palavras-chave}: arquitetura de \emph{software}. microsserviços. desenvolvimento. práticas. ferramentas.
\end{resumo}