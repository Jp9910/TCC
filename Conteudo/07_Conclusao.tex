\chapter{Conclusão}\label{chapter-conclusao}

Como pôde ser constatado, a escolha da arquitetura de uma aplicação não é uma decisão simples. Assim como quase tudo na computação, trata-se de \emph{tradeoffs}, e para determinar se a arquitetura a ser escolhida é adequada, precisa-se entender e contextualizar seus benefícios, riscos, desvantagens e desafios. Apesar de não existir uma definição formal para a arquitetura de microsserviços, há muitas características que a diferencia de outras abordagens arquiteturais, tais como a componentização, a evolução, e a complexidade.

Foi observada uma ampla concordância entre pesquisadores e praticantes de microsserviços acerca do que é comum, do que é bem-visto, do que é considerado um anti-padrão, e de quais são os desafios no desenvolvimento de aplicações com arquitetura de microsserviços, assuntos os quais foram contextualizados e discutidos neste trabalho. Também foi descoberto um ponto em que há mais espaço para discussão e pesquisa - a prática de se começar uma arquitetura de microsserviços por uma arquitetura monolítica até que a aplicação e seus domínios já estejam bem definidos -, pois foi observado certo nível de discordância entre os autores das bibliográfias revisadas sobre o que é ou não necessário para sustentar uma arquitetura de microsserviços desde o início do desenvolvimento da aplicação, e quais seriam as razões para se adotar ou não essa arquitetura.

A proposta de uma combinação de ferramentas que podem ser usadas no desenvolvimento de aplicações com arquitetura de microsserviços e suas contextualizações ficaram adiadas para serem exploradas na próxima etapa deste trabalho. Também ficaram para a próxima etapa o desenvolvimento de uma aplicação exemplar com arquitetura de microsserviços usando a combinação proposta e a discussão acerca dos pontos positivos e negativos observados no uso dessas ferramentas.

% Foram apresentadas as características mencionadas como imprescindíveis por pesquisadores e praticantes de microsserviços, assim como as que aparecem com mais frequência na literatura. 

% As conclusões constituem a parte final do texto, na qual se apresentam as considerações finais sobre o assunto, se os objetivos foram alcançados, o que se descobriu, quais outras questões surgiram a partir dos resultados e se as hipóteses se confirmaram ou não. Vale lembrar que nenhum trabalho de pesquisa encerra um tema ou problema, por isso, evite fazer afirmações redutoras ou definitivas.

% ?Comentar sobre o objetivo de Expor uma boa quantidade de ferramentas tal que supra as necessidades mais frequentes de uma arquitetura de microsserviços, com algumas alternativas.