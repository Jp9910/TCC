\chapter{Conclusão}\label{chapter-conclusao}

Como pôde ser constatado, apesar de não existir uma definição formal para a arquitetura de microsserviços, há muitas características que a diferencia de outras abordagens arquiteturais, tais como a componentização, a evolução, e a complexidade. Também foi observada uma ampla concordância entre pesquisadores e praticantes de microsserviços acerca do que é comum, do que é bem-visto, do que é considerado um anti-padrão, e de quais são os desafios no desenvolvimento de aplicações com arquitetura de microsserviços, assuntos os quais foram contextualizados e discutidos neste trabalho. 

Foi descoberto um ponto em que há mais espaço para discussão e pesquisa - a técnica de se começar uma arquitetura de microsserviços por uma arquitetura monolítica até que a aplicação e seus domínios já estejam bem definidos -, pois foi observado certo nível de discordância entre os autores das bibliográfias revisadas sobre o que é ou não necessário para sustentar uma arquitetura de microsserviços desde o início do desenvolvimento da aplicação, e quais seriam as razões para se adotar ou não essa arquitetura.

Também foram apontadas ferramentas que podem ser usadas no desenvolvimento de aplicações com arquitetura de microsserviços, embora suas contextualizações e detalhes ficaram adiados para serem discutidos na próxima etapa deste trabalho. Também ficou para a próxima etapa a proposição de uma combinação dessas ferramentas, o desenvolvimento de uma aplicação exemplar com arquitetura de microsserviços a utilizando, e a discussão acerca dos pontos positivos e negativos constatados durante o desenvolvimento.

% Foram apresentadas as características mencionadas como imprescindíveis por pesquisadores e praticantes de microsserviços, assim como as que aparecem com mais frequência na literatura. 

% As conclusões constituem a parte final do texto, na qual se apresentam as considerações finais sobre o assunto, se os objetivos foram alcançados, o que se descobriu, quais outras questões surgiram a partir dos resultados e se as hipóteses se confirmaram ou não. Vale lembrar que nenhum trabalho de pesquisa encerra um tema ou problema, por isso, evite fazer afirmações redutoras ou definitivas.