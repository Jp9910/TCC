\chapter{Fundamentação teórica}\label{cap_exemplos}

\chapterprecis{Introdução sobre as arquiteturas de monolito e de microserviços. Trabalhos relacionados.}\index{sinopse de capítulo}

\section{Os monolitos}

Aplicações monolíticas são aplicações (não compostas?) que possuem as camadas de acesso aos dados, de regras de negócios, e de interface de usuário em um único programa em uma única plataforma. Os monolitos são autocontidos e totalmente independentes de outras aplicações. Eles são feitos não para uma tarefa em particular, mas sim para serem responsáveis por todo o processo para completar determinada função. Em outras palavras, as aplicações monolíticas não têm modularidade. Elas podem ser organizadas das mais variadas formas, e fazer uso de padrões arquiteturais, mas são limitadas em muitos outros aspectos.

\subsection{Benefícios}

O maior e melhor benefício da arquitetura monolítica é sua simplicidade. Uma aplicação simples é uma aplicação fácilmente entendida pelos seus desenvolvedores, o que melhora sua manutenibilidade. Para aplicações com um domínio simples, como um e-commerce de calçados por exemplo, optar por uma arquitetura complexa como a de microserviços significaria adicionar uma enorme complexidade - provavelmente desnecessária - em seu desenvolvimento e infraestrutura.

Outra vantagem dos monolitos é sua facilidade de construção, tanto em relação a sua infraestrutura quanto ao seu desenvolvimento. Dentre todos os tipos de arquitetura, os monolitos têm o tipo de infraestrutura mais fácil de se construir, e além disso, nos monolitos geralmente não é necessário haver comunicação entre diferentes serviços ou máquinas, então os desenvolvedores não precisarão se preocupar com a complexidade que acompanha essa comunicação.

Até certo tamanho, são fáceis de manter porque são fáceis de serem entendidos. Porém, depois de crescer excessivamente, um monolito pode se tornar um emaranhado complexo de funcionalidades que são difíceis de diferenciar, de separar, e de manter. E então começam a surgir as limitações deles...

\subsection{Limitações}

Crescimento, escalamento, manutenção, reutilização, flexibilidade (elaborar melhor)

This had several limitations such as inflexibility, lack of reliability, difficulty scaling, slow development, and so on. It was to bypass these issues that microservices architecture was created.

Alguns problemas da solução monolítica:
- A necessidade de \emph{buildar} toda a aplicação, mesmo as partes em que não houveram mudanças, a cada \emph{deploy}.
- Falhas relativamente pequenas podem prejudicar toda a aplicação, mesmo as partes que não tiveram relação com a falha.
- As escolhas de tecnologias são mais limitadas. Um projeto **tende** a usar apenas 1 solução.

% ---
\section{O que são microserviços}
% ---

Aplicações com uma arquitetura de microserviços são separadas em partes pequenas, chamadas de microserviços, que são classificadas e se comunicam por meio de uma rede. Microserviços oferecem capacidades de negócio ou de plataforma, tratando um aspecto em particular da aplicação. Eles se comunicam por meio de APIs bem definidas, contratos de dados, e configurações. O "micro" em microserviços faz referência não ao tamanho do serviço, mas sim ao seu escopo de funcionalidade. Eles oferecem apenas uma determinada funcionalidade, tornando-se especialistas nela. Assim sendo, microserviços não necessariamente devem ser pequenos em tamanho, mas fazem apenas uma tarefa e a fazem eficientemente. 
 
Sendo especialistas em apenas uma tarefa, microserviços têm propriedades e comportamentos que os diferenciam de outras arquiteturas orientadas a serviços, que serão melhor discutidos no próximo capítulo.


\section{Trabalhos relacionados}

\section*{"Microservices, IoT and Azure", por Bob Familiar - capítulo 2: "What is a microservice"}

O capítulo 2 do livro de Bob Familiar descreve o que é um microserviço, suas características e implicações, benefícios, e desafios. 

"Microservices do one thing and they do it well". Como é explicado por \citeonline{Familiar2015} , microserviços representam business capabilities definidos usando o design orientado a domínio, são testados a cada passo do \emph{pipeline} de \emph{deploy}, e lançados por meio de automação, como serviços independentes, isolados, altamente escaláveis e resilientes em uma infraestrutura em núvem distribuída. Pertecem a um time único de desenvolvedores, que trata o desenvolvimento do microserviço como um produto, entregando software de alta qualidade em um processo rápido e iterativo com envolvimento do cliente e satisfação como métrica de sucesso.

\section*{"A Systematic Mapping Study on Microservices Architecture in DevOps", por Waseem, M., Liang, P. e Shahin, M.}

Esse trabalho tem o objetivo de sistematicamente identificar, analisar, e classificar a literatura sobre microserviços em DevOps.

Inicialmente o leitor é contextualizado no mundo dos microserviços e a cultura DevOps. Os autores usam a metodologia de pesquisa de um estudo de mapeamento sistemático da literatura publicada entre Janeiro de 2009 e Julho de 2018. Após selecionados 47 estudos, é feita a classificação deles de acordo com os critérios definidos pelos autores, e então é feita a discussão sobre os resultados obtidos - são expostos a quantidade de estudos sobre determinados tópicos em microserviços, problemas e soluções, desafios, métodos de descrição, design patterns, benefícios, suporte a ferramentas, domínios, e implicações para pesquisadores e praticantes.

the key results are: (1) Three themes on the research on MSA in DevOps are “microservices development and operations in DevOps”, “approaches and tool support for MSA based systems in DevOps”, and “MSA migration experiences in DevOps”. (2) 24 problems with their solutions regarding implementing MSA in DevOps are identified. (3) MSA is mainly described by using boxes and lines. (4) Most of the quality attributes are positively affected when employing MSA in DevOps. (5) 50 tools that support building MSA based systems in DevOps are collected. (6) The combination of MSA and DevOps has been applied in a wide range of application domains. Conclusions: The results and findings will benefit researchers and practitioners to conduct further research and bring more dedicated solutions for the issues of MSA in DevOps.


Comparar cada trabalho com o meu trabalho. Coisas que eles não abordam e que eu abordo