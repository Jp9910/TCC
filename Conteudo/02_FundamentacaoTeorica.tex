\chapter{Fundamentação teórica}\label{chapter-fundamentacao}

\chapterprecis{Este capítulo aprensenta uma introdução sobre as arquiteturas monolítica e de microsserviços, e investiga trabalhos relacionados.}\index{sinopse de capítulo}

\section{As aplicações monolíticas}

Aplicações monolíticas, também chamadas de monólitos, são aplicações que possuem as camadas de acesso aos dados, de regras de negócios, e de interface de usuário em um único programa em uma única plataforma. Os monólitos são autocontidos e totalmente independentes de outras aplicações. Eles são feitos não para uma tarefa em particular, mas sim para serem responsáveis por todo o processo para completar determinada função. Em outras palavras, as aplicações monolíticas têm problema de modularidade. Elas podem ser organizadas das mais variadas formas e fazer uso de padrões arquiteturais, mas são limitadas em muitos outros aspectos, citados na \autoref{subsection-monolitos-limitacoes}. \cite{wiki_monolithic_2022}

\subsection{Benefícios}

O maior e melhor benefício da arquitetura monolítica é sua simplicidade. Os monólitos são simples para desenvolver, para implantar, e para escalar. Ademais, uma aplicação simples é uma aplicação fácilmente entendida pelos seus desenvolvedores, fato que por sí só já melhora sua manutenibilidade.

Outra vantagem dos monólitos é sua facilidade de construção em relação a sua infraestrutura. Por não possuirem dependências com outras aplicações e nem precisarem se dedicar a comunicação externa, os monólitos têm uma infraestrutura fácil de se estruturar.
%e além disso, neles geralmente não é necessário haver comunicação entre diferentes serviços ou máquinas, então os desenvolvedores não precisarão se preocupar com a complexidade que acompanha essa comunicação.

Até certo tamanho, as aplicações monolíticas são fáceis de manter porque são fáceis de serem entendidas. Porém, depois que a aplicação ou o time cresce muito, um monólito pode se tornar um emaranhado complexo de funcionalidades que são difíceis de diferenciar, de separar, e de manter. E então começam a surgir as limitações deles... \cite{microservicesIO_monolithic_architecture}

\subsection{Limitações}\label{subsection-monolitos-limitacoes}

As limitações das aplicações monolíticas incluem:

\subsubsection*{Confiabilidade}


\subsubsection*{Escalabilidade}


\subsubsection*{Reutilização}

\subsubsection*{Crescimento, velocidade de desenvolvimento, e manutenção}
Depois de chegar num certo tamanho, torna-se muito difícil desenvolver funcionalidades novas, ou mesmo prover manutenção às já existentes. Padrões de organização podem amenizar a situação, mas não eliminam o problema.


\subsubsection*{Implantação}
Necessidade de compilar toda a aplicação, mesmo as partes em que não houve mudanças, a cada implantação.

\subsubsection*{Resiliência}
Falhas relativamente pequenas podem prejudicar toda a aplicação, mesmo as partes que não tiveram relação com a falha.

\subsubsection*{Flexibilidade}
As escolhas de tecnologias são mais limitadas. Um projeto tende a usar apenas 1 solução devido a problemas de complexidade ou compatibilidade que podem surgir ao usar mais.

% ---
\section{Os microsserviços}
% ---

\begin{figure}[htb]
	\caption{\label{figura_arquitetura_microsservicos}Aplicação com arquitetura de microsserviços}
	\begin{center}
	    \includegraphics[scale=0.5]{Imagens/microservice_architecture.png}
	\end{center}
	\legend{Fonte: \citeonline{oracle_microservices}}
\end{figure}

Microsserviços é uma abordagem de arquitetura de software. Aplicações com uma arquitetura de microsserviços são separadas em partes, chamadas de microsserviços, que são classificadas e se comunicam por meio de uma rede. Microsserviços oferecem capacidades de negócio (funcionalidades relacionadas às regras de negócio da aplicação) ou capacidades de plataforma (funcionalidades relacionadas ao ambiente de execução da aplicação), tratando um aspecto em particular da aplicação. Eles se comunicam por meio de APIs bem definidas, contratos de dados, e configurações. O "micro" em microsserviços faz referência não ao tamanho do serviço, mas sim ao seu escopo de funcionalidade. Eles oferecem apenas uma determinada funcionalidade, tornando-se especialistas nela. Assim sendo, microsserviços não necessariamente devem ser pequenos em tamanho, mas fazem apenas uma tarefa e a fazem eficientemente. \cite{Familiar2015}
 
Sendo especialistas em apenas uma tarefa, microsserviços têm características e comportamentos que os diferenciam de outras arquiteturas orientadas a serviços, os quais serão discutidos no \autoref{chapter-caracteristicas}.

A \autoref{figura_arquitetura_microsservicos} exemplifica uma aplicação com arquitetura de microserviços. Inicialmente os usuários da aplicação (camada \emph{Clients}) fazem requisições à API para obter as informações desejadas. O \emph{API Gateway} - que é reponsável por gerenciar as chamadas aos microsserviços e será melhor discutido no \autoref{boas-praticas-api-gateway} - fará as devidas requisições para os devidos microsserviços (localizados na camada \emph{Container Orchestration}). Esses microsserviços então buscarão a informação necessária no devido banco de dados (camada \emph{Data Store}).

% A ilustração mostra as camadas da arquitetura de microsserviços. A primeira camada é a camada do cliente, que contém: os computadores, laptops e dispositivos móveis. A segunda camada é a camada do gateway de API, que redireciona as solicitações do cliente para os microsserviços apropriados. A terceira camada é a camada de orquestração de contêineres, com todos os microsserviços agrupados dentro de uma orquestração de contêineres. Cada um dos microsserviços está em contêiner. Eles se comunicam com os clientes por meio do gateway de API. A quarta camada é a camada de armazenamento de dados. Cada um dos microsserviços em contêiner que implementam persistência se comunica a apenas um armazenamento de dados. Os armazenamentos de dados exibidos são NoSQL e SQL.


% Microsserviços são uma abordagem arquitetônica e organizacional do desenvolvimento de software na qual o software consiste em pequenos serviços independentes que se comunicam usando APIs bem definidas. Esses serviços pertecem a pequenas equipes autossuficientes.

% A arquitetura de microsserviços (AMS) está ganhando força no desenvolvimento e entrega de aplicações de software como um conjunto de pequenos serviços granulares que podem ser integrados por meio de mecanismos de comunicação leve, normalmente APIs RESTful [10]. Microsserviços são componentes pequenos e facilmente entendíveis que possuem capacidades de negócio no meio dos serviços [11]. Esses serviços podem ser escalados independentemente (já que são desacoplados) pela implementação de \texttt{stacks} de tecnologias diferentes [2]. Muitos pesquisadores e praticantes dizem que AMS é uma evolução da Arquitetura orientada a serviços (AOS), como visto no contexto de serviços independentes/auto-suficientes e de natureza leve [12]. Por outro lado, AMS pode ser diferenciada da AOS em termos de compartilhamento de componentes, comunicação de serviços, mediação de serviços, e acesso remoto aos serviços [13]. (Bar, f., 2018, tradução nossa). \cite{WASEEM2020110798}
% % AOS é construida com base na ideia de compartilhar o máximo possível, enquanto AMS, o mínimo possível [13, 14]. AMS usa um estilo coreografico para comunicação inter-serviços, enquanto AOS aplica um estilo de orquestração para coordenação de serviços. Para mediação de serviços, AMS usa a camada de API que atua como uma fachada para o serviço, enquanto AOS adota o conceito de um \texttt{middleware} mensageiro para coodenação de serviços. Além disso, AMS em grande parte depende do protocolo REST e mensageria simples como protocolo de acesso remoto ao serviço; entretanto, AOS consegue lidar com diferentes tipos de protocolo de acesso remoto, incluindo mensageria simples para acessar serviços remotos [13].

\subsection{Tipos de microsserviços}

\subsubsection{Serviço de dados (data service)}
Tipo de serviço mais baixo-nível. Responsável por receber e tratar dados, assim fornecendo acesso a determinado domínio e suas regras. Quando um serviço de dados realiza apenas operações relacionadas a um determinado domínio da aplicação, ele também é chamado de serviço de domínio.

\subsubsection{Serviço de negócio (business service)}
Em determinados momentos as operações precisam de mais de um modelo do domínio para serem representadas em um serviço. Assim, os serviços de negócio agregam dados e oferecem operações mais complexas. Eles englobam vários serviços de domínio e proveem uma funcionalidade do negócio de nível mais alto, podendo também encapsular domínios relacionados. Por exemplo, em um site de cursos online, um serviço de negócio poderia prover uma funcionalidade chamada "Matricular Aluno", que envolveria as operações de inserir aluno no serviço de cursos, inserir aluno no serviço de pagamento, e inserir aluno no serviço de gamificação.

\subsubsection{Serviço de tradução (translation service)}
Um serviço de tradução é um intermediário entre a aplicação e um recurso externo, provendo uma forma de acessar esse recurso. No caso desse serviço externo sofrer mudanças, pode-se realizar as alterações consequentemente necessárias em apenas um lugar, nesse serviço de tradução. Por exemplo, a aplicação pode consumir uma API externa por meio do serviço de tradução, pedindo para que ele faça uma requisição para essa API, e então recebendo a resposta. % App --> translation service --> Api externa

\subsubsection{Serviço de ponta (edge service)}
É um serviço que serve diretamente ao cliente, sendo customizado para atender necessidades específicas desse cliente. Por exemplo, pode existir um serviço de ponta para clientes móveis e outro serviço de ponta para clientes web.

\section{Trabalhos relacionados}

\subsection{{Microservices, IoT and Azure - capítulo 2: What is a microservice, por \citeonline{Familiar2015}}}

O capítulo 2 do livro de Bob Familiar descreve o que é um microsserviço, suas características e implicações, benefícios, e desafios. 

"Microservices do one thing and they do it well". Como é explicado por \citeonline{Familiar2015} , microsserviços representam business capabilities definidos usando o design orientado a domínio, são testados a cada passo do \emph{pipeline} de implantação, e lançados por meio de automação, como serviços independentes, isolados, altamente escaláveis e resilientes em uma infraestrutura em núvem distribuída. Pertecem a um time único de desenvolvedores, que trata o desenvolvimento do microsserviço como um produto, entregando software de alta qualidade em um processo rápido e iterativo com envolvimento do cliente e satisfação como métrica de sucesso.

Em contraste com o trabalho atual, \citeonline{Familiar2015} não aborda boas práticas, padrões e ferramentas usadas no desenvolvimento de microsserviços.

\subsection{{A Systematic Mapping Study on Microservices Architecture in DevOps, por \citeonline{WASEEM2020110798}}}

Esse trabalho tem o objetivo de sistematicamente identificar, analisar, e classificar a literatura sobre microsserviços em DevOps. Inicialmente o leitor é contextualizado no mundo dos microsserviços e a cultura DevOps. Os autores usam a metodologia de pesquisa de um estudo de mapeamento sistemático da literatura publicada entre Janeiro de 2009 e Julho de 2018. Após selecionados 47 estudos, é feita a classificação deles de acordo com os critérios definidos pelos autores, e então é feita a discussão sobre os resultados obtidos - são expostos a quantidade de estudos sobre determinados tópicos em microsserviços, problemas e soluções, desafios, métodos de descrição, design patterns, benefícios, suporte a ferramentas, domínios, e implicações para pesquisadores e praticantes.

Em contraste com o trabalho atual, \citeonline{WASEEM2020110798} não aborda as características dos microsserviços, mas também mapeia desafios enfrentados e soluções
empregadas.

% Os principais resultados são: (1) São identificados Três temas de pesquisa em AMS com DevOps “desenvolvimento e operações de microsserviços em DevOps”, “abordagens e suporte a ferramentas para sistemas baseados em AMS em DevOps”, e “Experiência de migração de AMS em DevOps”. (2) São identificados 24 problemas e apontadas suas respectivas soluções com respeito a implementação de microsserviços com DevOps. (3) A AMS é descrita princiapalmente usando caixas e linhas. (4) A maioria das qualidades da AMS são afetadas positivamente quando aplicadas com DevOps. (5) 50 ferramentas que suportam a construção de sistemas baseados em AMS são apontados. (6) A combinação da AMS e DevOps tem sido aplicada em uma ampla variedade de domínios de aplicações.

\subsection{Design, monitoring, and testing of microservices systems: The practitioners’ perspective, por \citeonline{design-monitoring-testing-waseem}}

Esse trabalho tem o objetivo de entender como sistemas de microsserviços são projetados, monitorados, e testados na indústria. Foi conduzida uma pesquisa relativamente grande que obteve 106 respostas e 6 entrevistas com praticantes de microsserviços. Os resultados obtidos identificam os desafios que os praticantes enfrentam e as soluções empregadas no projeto, monitoramento e teste de microsserviços. Também é feita uma discussão profunda sobre os resultados, da perspectiva dos praticantes, e sobre as implicações para pesquisadores e praticantes.

Em contraste com o atual trabalho, \citeonline{design-monitoring-testing-waseem} não abordam as características dos microsserviços.

% It identifies the challenges that practitioners face and the solutions employed by them when designing, monitoring, and testing microservices systems.

% It provides an in-depth discussion on the findings from the microservices practitioners’ perspective and implications for researchers and practitioners.

% (Comparar cada trabalho com o meu trabalho. Coisas que eles não abordam e que eu abordo)
