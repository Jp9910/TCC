\chapter{Aplicação exemplar}\label{chapter-aplicacao}

\chapterprecis{Este capítulo apresenta a aplicação exemplar desenvolvida.}\index{sinopse de capítulo}

\section{Domínio}
A aplicação desenvolvida trata-se de um sistema web de e-commerce. Nela, um cliente da loja pode buscar e comprar produtos, enquanto um administrador pode gerenciar produtos e usuários cadastrados, tudo a partir de uma interface de usuário em um navegador.

\section{Divisão dos microsserviços}

\subsection{Loja}
O microsserviço de loja trata da lógica de negócios relacionada a produtos e pedidos.

TODO: diagramas de classes, de pacotes, e de sequencia dos microsserviços

\subsection{Carrinho}
O microsserviço de carrinho trata da lógica de negócios relacionada ao carrinho e realização da compra.

\subsection{Usuários}
O microsserviço de usuários trata da lógica de negócios relacionada ao cadastro e autenticação de usuários.

\subsection{Serviço de ponta - Clientes}
O microsserviço de ponta de clientes é o serviço usado pelos clientes da loja para realizar todas as operações relevantes a eles a partir de uma interface de usuário, tal como ver e buscar produtos, adicionar produtos ao carrinho e realizar um pedido a partir de um carrinho.

\subsection{Serviço de ponta - Administração}
O microsserviço de ponta de administração é o serviço usado pelos administradores da loja para realizar todas as operações relevantes a eles a partir de uma interface de usuário, tal como gerenciar produtos e usuários.

\section{Práticas e ferramentas usadas}
Enumerar as práticas implementadas na aplicação