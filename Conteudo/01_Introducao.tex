\chapter{Introdução}\label{chapter-introducao}

In recent years, the rise of the internet and the ubiquity of mobile computing have made it necessary for application developers to design their applications focusing on a lightweight, self-contained component.

Developers need to deploy applications quickly and make changes to the application without a complete redeployment. This has led to a new development paradigm called "microservices," where an application is broken into a suite of small, independent units that perform their respective functions and communicate via APIs.

Although independent units, any number of these microservices may be pulled by the application to work together and achieve the desired results. \cite{noauthor_what_nodate}

%For the past several years, we have been developing standards and practices for team development of large, complex systems using a layered, monolithic architecture. This is reflected in how we organize into teams, structure our solutions and source code control systems, and package and release our software.

% ---
\section{Objetivos}\label{sec-objetivos}
% ---

% Esta seção descreve os objetivos do trabalho. Esta é a \autoref{sec-objetivos}. Veja os objetivos específicos em \autoref{sec-objetivos-especificos}.

\subsection{Objetivo geral}\label{sec-objetivo-geral}

Discutir, em alto nível, a arquitetura de microsserviços e suas características. Analisar os padrões, boas práticas, e soluções mais encontrados no desenvolvimento de aplicações que utilizam essa arquitetura. Modelar e implementar um exemplo de aplicação usando a arquitetura de microsserviços.
% Analisar e resumir o estado da arte em desenvolvimento de aplicações com arquitetura de microsserviços

\subsection{Objetivos específicos}\label{sec-objetivos-especificos}

% (TCC 1:)
- Caracterizar a arquitetura de microsserviços;

- Reunir padrões e práticas comuns na implementação de aplicações com arquitetura de microsserviços;

- Reunir soluções e ferramentas usadas na implementação de aplicações com arquitetura de microsserviços;

- Propor ideias e passos para como migrar do monolito para os microsserviços

% (TCC 2:)
Analisar a eficiência desses padrões e práticas.
% - Analisar/testar a eficiência desses padrões e práticas, por meio de (estudos de caso? análise da literatura? exemplos de empresas que as usam?);

Analisar a eficiência dessas soluções e ferramentas.
% - Analisar/testar a eficiência dessas soluções e ferramentas, por meio de (estudos de caso? análise da literatura? exemplos de empresas que as usam?);

- Propor uma combinação desses padrões e dessas ferramentas para a construção de uma aplicação com arquitetura de microsserviços.

\section{Metodologia}

Para caracterizar a arquitetura de microsserviços, foram pesquisados as seguintes termos nas bases científicas ScienceDirect, SpringerLink e GoogleScholar:

- (microservices or microservice) and pattern

- (microservices or microservice) and provision

metodologia quanto aos objetivos, quanto a execução. passo a passo que vai seguir durante o trabalho.

explicar como analisar/testar essa eficiencia (objetivos especificos)