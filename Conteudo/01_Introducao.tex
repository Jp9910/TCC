\chapter{Introdução}

In recent years, the rise of the internet and the ubiquity of mobile computing have made it necessary for application developers to design their applications focusing on a lightweight, self-contained component.

Developers need to deploy applications quickly and make changes to the application without a complete redeployment. This has led to a new development paradigm called "microservices," where an application is broken into a suite of small, independent units that perform their respective functions and communicate via APIs.

Although independent units, any number of these microservices may be pulled by the application to work together and achieve the desired results.

For the past several years, we have been developing standards and practices for team development of large, complex systems using a layered, monolithic architecture. This is reflected in how we organize into teams, structure our solutions and source code control systems, and package and release our software. 

% (Familiar, B. (2015). What Is a Microservice?. In: Microservices, IoT, and Azure. Apress, Berkeley, CA. https://doi.org/10.1007/978-1-4842-1275-2_2)

% ---
\section{Objetivos}\label{sec-objetivos}
% ---

Esta seção descreve os objetivos do trabalho. Esta é a
\autoref{sec-objetivos}. Veja os objetivos específicos em \autoref{sec-objetivos-especificos}.

\subsection{Objetivo geral}\label{sec-objetivo-geral}

Analisar e resumir o estado da arte em desenvolvimento de aplicações com arquitetura de microserviços, e elaborar sobre os padrões e soluções mais encontrados. (tcc2): Modelar/implementar um exemplo de aplicação com arquitetura de microserviços.

\subsection{Objetivos específicos}\label{sec-objetivos-especificos}

(TCC 1:)

- Resumir o estado da arte em desenvolvimento de aplicações com arquitetura de microserviços;

- Reunir padrões e práticas comuns na implementação de aplicações com arquitetura de microserviços;

- Reunir soluções e ferramentas usadas na implementação de aplicações com arquitetura de microserviços;

- (Propor um caminho de como migrar do monolito para os microserviços?)

(TCC 2:)

- Analisar/testar a eficiência desses padrões e práticas, por meio de (estudos de caso? análise da literatura? exemplos de empresas que as usam?);

- Analisar/testar a eficiência dessas soluções e ferramentas, por meio de (estudos de caso? análise da literatura? exemplos de empresas que as usam?);

- Propor uma combinação desses padrões e dessas ferramentas para a construção de um aplicativo com arquitetura de microserviços.