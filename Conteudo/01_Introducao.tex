\chapter{Introdução}\label{chapter-introducao}

O crescimento da internet e a onipresença da computação móvel tem mudado o jeito como \emph{software} é desenvolvido nos últimos tempos. Todos que têm contato com a área do desenvolvimento de \emph{software} provavelmente conhecem o termo \emph{SaaS (Software as a Service)}, ou \emph{software} como um serviço. Entretanto, essa expressão significa mais do que apenas um modelo de negócio. A tendência que tem-se observado na indústria do \emph{software} é a de oferecer \emph{software} não mais como um pacote completo e fechado, mas sim como um pacote flexível e em constante melhoria, o que implica na mudança do foco dos desenvolvedores para a criação de aplicações modulares, e que permitam que mudanças sejam desenvolvidas e implantadas rápida, fácil e independetemente \cite{CAOPLE, oracle_microservices}.

Essa mudança de foco implicou no surgimento de novas abordagens de arquitetura e organização de \emph{software}, e uma dessas tem ganho grande popularidade na indústria do \emph{software} por facilitar a criação de aplicações que são multilíngues, facilmente mantidas e implantadas, escaláveis, e altamente disponíveis. Inspirada na arquitetura orientada a serviços, ela se chama arquitetura de microsserviços, e é considerada por muitos profissionais da engenharia de software como a melhor maneira de arquitetar uma aplicação de \emph{software} como um serviço atualmente. Entretanto, como tudo na computação, há um \emph{trade-off} (uma troca), pois assim como há benefícios, também há desvantagens e desafios no emprego de uma arquitetura de microsserviços, os quais também serão discutidos neste trabalho \cite{middleware-microservices,design-monitoring-testing-waseem}.


% Many development teams have found the microservices architectural style to be a superior approach to a monolithic architecture. But other teams have found them to be a productivity-sapping burden. Like any architectural style, microservices bring costs and benefits. To make a sensible choice you have to understand these and apply them to your specific context. microservice tradeoffs - https://martinfowler.com/articles/microservice-trade-offs.html#summary

%Se você quiser projetar um aplicativo que seja multilíngue, facilmente escalável, fácil de manter e implantar, altamente disponível e que minimize falhas, use a arquitetura microservices para projetar e implantar um aplicativo em nuvem. Em uma arquitetura de microservices, cada microservice possui uma tarefa simples e se comunica com os clientes ou com outros microservices usando mecanismos de comunicação leves, como solicitações de API REST. \cite{oracle_microservices}

% O público alvo e os usuários desses softwares podem chegar a números imensos.

% In recent years, the rise of the internet and the ubiquity of mobile computing have made it necessary for application developers to design their applications focusing on a lightweight, self-contained component. Developers need to deploy applications quickly and make changes to the application without a complete redeployment. This has led to a new development paradigm called "microservices," where an application is broken into a suite of small, independent units that perform their respective functions and communicate via APIs. Although independent units, any number of these microservices may be pulled by the application to work together and achieve the desired results. \cite{middleware-microservices}

%For the past several years, we have been developing standards and practices for team development of large, complex systems using a layered, monolithic architecture. This is reflected in how we organize into teams, structure our solutions and source code control systems, and package and release our software.

% ---
\section{Objetivos}\label{sec-objetivos}
% ---

% Esta seção descreve os objetivos do trabalho. Esta é a \autoref{sec-objetivos}. Veja os objetivos específicos em \autoref{sec-objetivos-especificos}.

\subsection{Objetivo geral}\label{sec-objetivo-geral}

Analisar o desenvolvimento de aplicações com arquitetura de microsserviços.
% , discutindo boas práticas e ferramentas comumente usadas.
% Analisar e resumir o estado da arte em desenvolvimento de aplicações com arquitetura de microsserviços

\subsection{Objetivos específicos}\label{sec-objetivos-especificos}

% (TCC 1:)
- Caracterizar a arquitetura de microsserviços;
% - Apresentar? e caracterizar a arquitetura de microsserviços;

- Apresentar e discutir práticas comumente usadas no desenvolvimento de aplicações com arquitetura de microsserviços;
% - Apresentar? e discutir práticas comumente usadas no desenvolvimento de aplicações com arquitetura de microsserviços;

% - Analisar a viabilidade da aplicação a ser desenvolvida no TCC 2

% (TCC 2:)
% Analisar a eficiência dessas boas práticas;
% - Analisar/testar a eficiência desses padrões e práticas, por meio de (estudos de caso? análise da literatura? exemplos de empresas que as usam?);

% \textbf{Reescrever esse objetivo:}
% \sout{- Propor uma combinação de ferramentas para o desenvolvimento de aplicações com arquitetura de microsserviços;}

% \textbf{Pode ser reescrito como: }
% \sout{- Demonstrar uma combinação de ferramentas sendo usada em uma aplicação real com arquitetura de microsserviços, e possíveis alternativas}

% \textbf{Ou então:}
% \sout{- Propor uma combinação de ferramentas open-source ou gratuitas para o desenvolvimento de app com arq de microsserviços}

% Vai ser reescrito como:
- Apresentar ferramentas que são frequentemente usadas e que cumprem propósitos importantes em aplicações com arquitetura de microsserviços;
% Objetivo mais específico: Expor uma boa quantidade de ferramentas tal que supra as necessidades mais básicas/frequentes de uma arquitetura de microsserviços, com algumas alternativas.

- Contextualizar essas ferramentas, apontando os problemas que resolvem e necessidades que suprem, assim como seus pontos positivos e negativos;

- Desenvolver uma aplicação exemplar com arquitetura de microsserviços, usando uma combinação das ferramentas e práticas apresentadas;

% \textbf{Será que dá pra apontar pontos positivos e negativos de todas as ferramentas apresentadas? Talvez seja melhor apontar apenas os das ferramentas usadas..:}
% - Apontar pontos positivos e negativos das ferramentas usadas na aplicação exemplar.


\section{Metodologia}

Para o desenvolvimento deste trabalho, inicialmente foi feita uma pesquisa exploratória sobre a arquitetura de microsserviços, com o objetivo de ganhar maior familiaridade com o tema. Depois de definidos os objetivos iniciou-se uma pesquisa bibliográfica e os trabalhos mais relevantes foram filtrados e revisados. 

Como foi constatado que não existe uma definição formal para a arquitetura de microsserviços, para caracterizá-la e para reunir práticas foram extraídas dos trabalhos as características ou práticas que apareceram com frequência ou que foram mencionadas como imprescindíveis pelos autores. 
%As boas práticas que abordavam questões de código profundamente foram desconsideradas por serem extensas e específicas.
% apenas boas práticas relacionadas a infraestrutura foram consideradas?

Para apresentar e contextualizar as ferramentas frequentemente usadas em aplicações com arquitetura de microsserviços, uma nova pesquisa bibliográfica foi feita para conhecer as mais comumente usadas, entender como funcionam e os problemas que resolvem, assim como suas vantagens e limites. Após isso, e depois de decidido o domínio da aplicação exemplar, as ferramentas e práticas a serem usadas no desenvolvimento da aplicação foram escolhidas de acordo com as necessidades resultantes, tendo em mente os limites do contexto do desenvolvimento deste trabalho. 
%Durante o desenvolvimento, foram reunidos pontos positivos e negativos observados no uso das ferramentas.  
 

% (metodologia quanto aos objetivos, quanto a execução. passo a passo que vai seguir durante o trabalho. explicar como analisar/testar essa eficiencia (objetivos especificos))

% pesquisa bibliográfica refere-se ao ato de reunir os materiais
% a revisão bibliográfica refere-se ao ato de estudar os materiais e extrair aquilo que interessa