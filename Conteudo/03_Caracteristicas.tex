\chapter{Características}\label{chapter-caracteristicas}

\section{Propriedades}

\subsection{Autonomia e Isolamento}
Autonomia e isolamento significa que microsserviços são unidades auto-contidas de funcionalidade com dependências de outros serviços fracamente acopladas e são projetados, desenvolvidos, testados e lançados independentemente. \cite{Familiar2015}. O termo autônomo pode ser definido como - existe ou é capaz de existir independetemente das outras partes. O termo isolado, como - separado das outras partes.

\subsection{Elasticidade, resiliência, e responsividade}

Microsserviços são reusados entre muitas soluções diferentes e portanto devem ser escaláveis de acordo com o uso. Devem ser resilientes, isso é, ser tolerantes a falhas e ter um tempo de recuperação razoável quando algo der errado. Além disso, devem ser responsivos, tendo um desempenho razoável de acordo com o uso. \cite{Familiar2015} O termo elástico pode ser definido como - capaz de retornar ao tamanho/formato original depois de ser esticado, comprimido ou expandido. O termo resiliente, como - resistente às mudanças negativas. O termo responsivo, como - Rápido em responder e reagir.

\subsection{Orientação-a-mensagens e programabilidade}

Microsserviços dependem de APIs e contratos de dados para definir como interagir com o serviço. A API define um conjunto de endpoints acessíveis por rede, e o contrato de dados define a estrutura da mensagem que é enviada ou retornada. \cite{Familiar2015}. O termo orientado-a-mensagens pode ser definido como - Software que conecta sistemas separados em uma rede, carregando e distribuindo mensagens entre eles. O termo programável, como - Obedece a um plano de tarefas que são executadas para alcançar um objetivo específico.

\subsection{Configurabilidade}

Microsserviços devem provêr mais do que apenas uma API e um contrato de dados. Para que seja reusável e para que possa resolver as necessidades do sistema que o use, cada microsserviço tem níveis diferentes de configuração, e esta configuração pode ser feita de diferentes formas. \cite{Familiar2015}. O termo configurável pode ser definido como - Projetado ou adaptado para formar uma configuração ou para algum propósito.

\subsection {Automação}

O ciclo de vida de um microsserviço deve ser totalmente automatizado, desde o design até a implantação. O termo automatizado pode ser definido como - Funcionar sem precisar ser controlado diretamente.

\section{Vantagens}

\subsection{Evolução}

Quanto maior e mais antigo o software, mais difícil é de dar manutenção, e monolitos envelhecem com maior velocidade do que microsserviços. Entretanto, é possível migrar de um sistema monolítico para a arquitetura de microsserviços aos poucos, um serviço por vez, identificando capacidades de negócio, implementando-as como um microsserviço, e integrando com uso de padrões de baixo acoplamento. Ao longo do tempo, mais e mais funcionalidades podem ser separadas e implementadas como microsserviço, até que o núcleo da aplicação monolítica se transforme em apenas um outro serviço, ou um microsserviço. \cite{Familiar2015}

\subsection{Possibilidade de uso de diferentes ferramentas}

Cada microsserviço disponibiliza suas funcionalidades por meio de APIs e contratos de dados em uma rede. A comunicação independe da arquitetura que o microsserviço faz uso, então cada microsserviço pode escolher seu sistema operacional, linguagem e banco de dados. Isso é especialmente valioso para times com dificuldade de comunicação, pois cada time precisa apenas de conhecimento da arquitetura do microsserviço em que trabalha. \cite{Familiar2015}

\subsection{Alta velocidade}

Com um time responsável por cuidar do ciclo de desenvolvimento e sua automação, a velocidade com que microsserviços podem ser desenvolvidos é muito maior do que fazer o equivalente para uma solução monolítica. \cite{Familiar2015}

\subsection{Reusável e combinável}

Microsserviços são reusáveis por natureza. Eles são entidades independentes que provêm funcionalidades em um determinado escopo por meio de padrões de internet aberta. Para criar soluções para o usuário final, multiplos microsserviços podem ser combinados. \cite{Familiar2015}

\subsection{Flexibilidade no ambiente de execução}

A implantação de microsserviços é altamente dependente de sua automação. Para garantir flexibilidade de ambiente de execução, essa automação pode incluir configuração de cenários diferentes de uso, não apenas para produção, mas também para desenvolvimento e testagem, possibilitando que o microsserviço tenha o melhor desempenho em diversos cenários. Para tanto, é necessário o uso de ferramentas que configurem essa flexibilidade. \cite{Familiar2015}. Tais ferramentas serão melhor discutidas no \autoref{chapter-solucoes}.

\subsection{Flexibilidade na escolha de tecnologias}

Cada microsserviço pode ser desenvolvido usando uma linguagem de programação e estrutura que melhor se adapte ao problema que ele é projetado para resolver, o que oferece mais possibilidades de tecnologias para usar. \cite{oracle_microservices}

\subsection{Versionável e Substituível}

Com o controle completo dos cenários de implantação, é possível manter versões diferentes de um mesmo serviço rodando ao mesmo tempo, proporcionando retrocompatibilidade e fácil migração. Além disso, serviços podem ser atualizados ou mesmo substituidos sem ocasionar indisponibilidade do serviço. \cite{Familiar2015}


\section{Desafios}

However, developing applications in the microservices architecture presents three main challenges: (a) how to program systems that consists of a large number of services running in parallel and distributed over a cluster of computers; (b) how to reduce the communication overhead caused by executing a large number of small services; (c) how to support the flexible deployment of services to a network to achieve system load balance. \cite{CAOPLE}

\subsection{Comunicação}

cross-platform compatibility issues and inconsistent call standards issues in the process of development and call microservices. \cite{ZUO2020102878}

\subsection{[re]Organizaçao}

Organizar o sistema e o time para sustentar uma arquitetura de microsserviços é um grande desafio. Como explica \citeonline{Familiar2015}: 
\begin{citacao}
    If you are part of a command-and-control organization using a waterfall software project management approach, you will struggle because you are not oriented to high-velocity product development. If you lack a DevOps culture and there is no collaboration between development and operations to automate the deployment pipeline, you will struggle. \cite{Familiar2015}
\end{citacao}

\subsection{Plataforma}
Criar o ambiente de execução para microsserviços requer um grande investimento em infraestrutura dinâmica em \emph{data centers} dispersos para garantir maior disponibilidade. Se sua atual plataforma \emph{on-premises} não suporta automação, infraestrutura dinâmica, escalamento elástico e alta disponibilidade, deve-se considerar uma plataforma na núvem. \cite{Familiar2015}. Mais sobre soluções na núvem será discutido no \autoref{chapter-solucoes}.

\section{Identificação}

Domain-driven design (projeto orientado a domínio) é uma técnica bem consolidada e muito usada no desenvolvimento de software. Entretanto, para aplica-la em microsserviços, é necessário analisar onde cada peça desse padrão de projeto deve ficar. Em vez de projetar os modelos e os contextos limitados separando-os em camadas, deve-se juntar os contextos com seus respectivos modelos, e procurar por possíveis pontos de separação da aplicação - um lugar onde a linguagem muda, por exemplo. Isso resultaria em um ponto de partida para separar as partes e formar uma arquitetura de microsserviços. \cite{Familiar2015}

esse parágrafo não encaixa nas caracteristicas - desafios ?...

\section{Testes}

Assim como em qualquer aplicação, o teste é uma parte crucial do seu desenvolvimento. Escrever e testar código não muda muito entre as arquiteturas monolítica e de microsserviços, contudo, nos microserviços existem mais testes a serem executados. Não deve-se testar o microsserviço apenas antes de seu lançamento, mas sim em cada passo do \emph{pipeline} de implantação, sempre automatizando o máximo de etapas possível, para assim garantir uma entrega rápida de software de qualidade. \cite{Familiar2015}

\section{Descoberta}

Encontrar microsserviços em um ambiente distribuido pode ser feito de algumas maneiras diferentes. A informação pode ser armazenada diretamente no código, pode ser guardada e acessada em um arquivo, ou pode ser construido um microsserviço para encontrar outros microsserviços e disponibilizar suas localizações. Contudo, para prover detectabilidade como um serviço será necessário adquirir um produto de terceiros, integrar um projeto aberto, ou desenvolver sua própria solução. \cite{Familiar2015}
