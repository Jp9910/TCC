\chapter{Características, vantagens e desafios da arquitetura de microsserviços}\label{chapter-caracteristicas}

\chapterprecis{Este capítulo apresenta as características e as vantagens da arquitetura de microsserviços, assim como os riscos, desafios e desvantagens que as acompanham.}\index{sinopse de capítulo}

% \section{Propriedades dos microsserviços}

Em geral não existe uma definição formal do que a arquitetura de microsserviços tem ou não tem, entretanto as características aqui apresentadas são características comuns observadas em arquiteturas que se encaixam como microsserviços. Assim sendo, nem todas as arquiteturas de microsserviços terão essas características, apesar de ser esperado que a maioria delas estejam presentes \cite{martin-fowler-microservices}.

\section{Sistema distribuído}
A arquitetura de microsserviços forma naturalmente um sistema altamente distribuido, o que implica em alguns comportamentos e características. Uma delas é que qualquer chamada a um serviço está sujeita a falhas, portanto microsserviços devem ser projetados para serem resilientes, isso é, tolerantes a falhas e ter um tempo de recuperação razoável quando algum problema inevitavelmente acontecer \cite{Familiar2015}.

Um benefício proveniente dessa distribuição é que microsserviços podem ser usados em soluções e cenários de uso diferentes, contudo, para tanto devem ser escaláveis, responsivos e configuráveis, para assim alcançar um bom desempenho independente do cenário de uso. Outro benefício dessa distribuição é a autonomia e o isolamento, o que significa que microsserviços são unidades auto-contidas de funcionalidade, com dependências de outros serviços fracamente acopladas, então podem ser projetados, desenvolvidos, testados e implantados independentemente de outros serviços \cite{martin-fowler-microservices,Familiar2015}.

Um desafio advindo dessa distribuição é que a comunicação é complexa e falível, geralmente dependendo de APIs e contratos de dados para definir como os microsserviços interagem, portanto eles são orientados-a-mensagens. Essa comunicação é melhor discutida na \autoref{secao-comunicacao} e \autoref{subsecao-contratos-de-dados} \cite{Familiar2015}. 

% A API define um conjunto de \emph{endpoints} acessíveis por rede, e o contrato de dados define a estrutura da mensagem que é enviada ou retornada \cite{Familiar2015}.

% \section{Vantagens}

\section{Flexibilidade na escolha de ferramentas}

Como mencionado anteriormente, cada microsserviço disponibiliza suas funcionalidades por meio de APIs e contratos de dados em uma rede. Usando esse meio, a comunicação independe da arquitetura que o microsserviço faz uso, o que possibilita que cada microsserviço escolha seu sistema operacional, linguagem, serviços de apoio e demais ferramentas necessárias. Além disso, os microsserviços podem ser desenvolvidos usando uma linguagem de programação e estrutura que melhor se adapte ao problema que ele é projetado para resolver, possuindo mais flexibilidade. Entretanto, é importante notar que esse benefício pode resultar em um desafio, pois usar muitas ferramentas diferentes numa mesma aplicação aumenta exponencialmente sua complexidade de desenvolvimento e manutenção \cite{oracle_microservices, Familiar2015}.

\section{Alta velocidade de desenvolvimento}

Ter um time independente responsável por cuidar do ciclo de desenvolvimento e sua automação permite uma alta velocidade de desenvolvimento para os microsserviços, muito maior do que fazendo o equivalente para uma solução monolítica \cite{Familiar2015}.

\section{Componentização}\label{secao-componentizacao}

Há muito tempo na indústria do \emph{software} deseja-se construir sistemas apenas juntando componentes, assim como se faz no mundo físico. Na computação, um componente é definido como uma unidade de \emph{software} que é atualizável e substituível independentemente. Apesar de ser muito comum o uso de pacotes e bibliotecas (padrão de projeto conhecido como \emph{sidecar}), que podem ser considerados componentes, há maneiras diferentes de se componentizar \emph{software} que são características dos microsserviços. Os microsserviços também podem utilizar pacotes e bibliotecas como componentes, contudo nessa arquitetura a maneira principal e mais eficiente para componentizar o \emph{software} é justamente dividí-lo em microsserviços. Entretanto, essa divisão não é uma tarefa simples - pelo contrário, definir adequadamente os limites dos microsserviços é um dos desafios mais complexos e importantes desta arquitetura, mas pode ser facilitado pelo uso de abordagens bem consolidadadas de design de \emph{software}, como o \emph{Domain-Driven Design} (DDD), discutido na \autoref{section-ddd} \cite{martin-fowler-microservices}.

Como mencionado acima, uma maneira de se alcançar certo nível de componentização em uma aplicação é pelo uso de múltiplas bibliotecas como componentes em um único processo, porém nesse caso uma mudança em qualquer desses componentes resulta na necessidade de reimplantar toda a aplicação. Por outro lado, se essa mesma aplicação é decomposta em múltiplos serviços, é provável que uma mudança em um serviço só obrigaria a reimplantação do mesmo serviço, assim tendo-se uma implantação mais simples e rápida. Ademais, por ter um alto nível de componentização e independência, os microsserviços são reusáveis por natureza e portanto torna-se mais fácil criar soluções e produtos por meio da combinação de multiplos microsserviços \cite{martin-fowler-microservices,Familiar2015}.

Contudo, usar serviços dessa forma também traz algumas desvantagens. Uma delas é que será necessário considerar como mudanças em um serviço podem afetar seus consumidores. A abordagem tradicional para resolver esse problema é o uso de versionamento no serviço, entretanto essa prática não é bem-vista no mundo dos microsserviços. A melhor solução é projetar serviços para serem o mais tolerante possíveis a mudanças nos serviços a que consomem. Outra desvantagem é que a comunicação remota é muito mais complexa e custosa, portanto o método de comunicação escolhido deve ser implementado de modo flexível. Ademais, realocar responsabilidades entre componentes é mais difícil quando se trata de processos diferentes \cite{martin-fowler-microservices}.

\section{Portabilidade}\label{sec:portabilidade} % Flexibilidade no ambiente de execução

A portabilidade de uma aplicação diz respeito a quão facilmente ela pode ser executada em diferentes ambientes de execução com variadas configurações de plataforma e infraestrutura. Ela não é inerente aos microsserviços, mas é facilitada por terem escopo e tamanho limitados. No entanto, faz-se necessário o uso de técnicas que aumentem essa portabilidade, como por meio de \hyperref[secao-conteinerizacao]{conteinerização}, automação e externalização de configurações, para que assim os microsserviços possam ser implantados de forma fácil e eficiente em ambientes de execução variados. \cite{Familiar2015}.
% \autoref{secao-externalizacao-de-configuracoes}

\section{Versionável e substituível}

Apesar do versionamento de microsserviços não ser recomendado por dificultar a operação e o entendimento deles, quando necessário é possível manter versões diferentes de um mesmo serviço executando ao mesmo tempo, assim proporcionando retrocompatibilidade e um processo de migração mais suave. Além disso, serviços podem ser atualizados ou mesmo substituidos sem ocasionar indisponibilidade do serviço, pelo uso de ferramentas de provisionamento e técnicas de implantação apropriadas. \cite{Familiar2015}.
% como a \emph{blue-green deployment} 

\section{Menor erosão de \emph{software}}

A erosão de \emph{software} é inevitável em qualquer aplicação - conforme um sistema cresce e envelhece, é natural que torne-se mais difícil dar-lhe manutenção. Porém, com componentes modularizados e organizados adequadamente, uma aplicação com arquitetura de microsserviços tende a erodir muito mais lentamente e crescer mais horizontalmente (em vez de verticalmente) do que uma não modularizada, favorecendo uma vida útil mais longa da aplicação.

% É possível decompor uma aplicação monolítica em uma arquitetura de microsserviços - tema de pesquisa muito recorrente na área de arquitetura de \emph{software} - mas esse processo não faz parte do escopo deste trabalho.

% Entretanto, é possível migrar de um sistema monolítico para a arquitetura de microsserviços aos poucos, um serviço por vez, identificando capacidades de negócio, implementando-as como um microsserviço, e integrando com uso de padrões de baixo acoplamento. Ao longo do tempo, mais e mais funcionalidades podem ser separadas e implementadas como microsserviço, até que o núcleo da aplicação monolítica se transforme em apenas um outro serviço, ou um microsserviço. \cite{Familiar2015}

\section{Complexidade e desafios}
% Capítulo 1 do livro \cite{livro-building-microservices} tem uma sessão Microservice Pain Points com muita informação boa

O uso da arquitetura de microsserviços implica num grande aumento de complexidade não apenas na infraestrutura, mas também na segurança da aplicação e em muitas etapas do ciclo de desenvolvimento do \emph{software}, incluindo o desenvolvimento, testes, monitoramento e \emph{debug}. Além disso, o uso de diversas ferramentas diferentes, o que é comum nessa arquitetura, pode requerer o envolvimento de mais desenvolvedores e com mais experiência, também podendo ocasionar problemas por inexperiência deles. \cite{top10-microservices-challenges,martin-fowler-microservice-tradeoffs,livro-building-microservices}.

Ademais, um dos principais desafios no desenvolvimento de uma aplicação com arquitetura de microsserviços é a definição adequada dos limites de cada microsserviço, que pode ser facilitado pelo uso de abordagens bem consolidadadas de \emph{design} de \emph{software}, como o DDD. Outros desafios são: complexidade de projeto, complexidade operacional, consistência de dados, comunicação e manutenção. De acordo com \citeonline{CAOPLE}, os \textbf{três grandes desafios} do desenvolvimento de aplicações com arquitetura de microsserviços são (1) Como programar sistemas que consistem de um grande número de serviços executando em paralelo e distribuídos em um conjunto de máquinas, (2) Como reduzir a sobrecarga de comunicação causada pela execução de grandes números de pequenos serviços e (3) Como sustentar a implantação flexível de serviços em uma rede para conseguir realizar o balanceamento de carga. \cite{martin-fowler-monolith-first}.

\subsection*{Teorema CAP - Disponibilidade ou Consistência}\label{teorema-cap}
CAP é um acrônimo para \emph{Consistency, Availability and Partition tolerance} (Consistencia, Disponibilidade e Tolerância à partição), e o teorema CAP prova que é impossível, em um sistema distribuido, se ter consistência estrita de dados, disponibilidade contínua de todos os nós e tolerância à falhas de comunicação entre os nós ao mesmo tempo.
% Só é possível então se ter 2 desses - AP, CP ou CA.

Um sistema distribuido que se mantém completamente disponível e tolera partições (AP) precisará lidar com a consistência eventual (em vez de estrita) dos dados durante uma falha. Um sistema distribuido que mantém consistência estrita dos dados e tolera partições (CP) precisará ficar indisponível enquanto se recupera de uma falha. Já um sistema que mantém consistência estrita dos dados e continua completamente disponível durante uma falha, na verdade não pode ser um sistema distribuido, pois sendo intolerante à partições, não poderia estar se comunicando por meio de uma rede. Assim, não se pode abandonar a tolerância à partições em sistemas distribuídos, reduzindo a escolha a consistência estrita ou disponibilidade contínua. Como toda aplicação com arquitetura de microsserviços naturalmente forma um sistema distribuido, é importante ter isso em mente ao escolher um modelo de gerenciamento de dados. \cite{teorema-cap-ibm,livro-building-microservices}

Percebe-se, então, que existem benefícios, desvantagens e desafios no desenvolvimento e manutenção de aplicações com arquitetura de microsserviços, normalmente só conseguindo ser compensados em aplicações mais complexas, que se beneficiam melhor dessa arquitetura. Portanto, para determinar se adotar essa arquitetura é uma escolha sábia é necessário entender esses benefícios, desvantagens e desafios e aplicá-los ao contexto específico da aplicação e dos desenvolvedores \cite{martin-fowler-monolith-first}.

% microservices impose a cost on productivity that can only be made up for in more complex systems. So if you can manage your system's complexity with a monolithic architecture then you shouldn't be using microservices. \cite{martin-fowler-microservice-tradeoffs}

% De acordo com \citeonline{design-monitoring-testing-waseem}, mais pesquisas são necessárias para lidar com a complexidade dos microsserviços no nível de projeto, de monitoramento, e de testes, desafios para qual não há soluções dedicadas.

%how to program systems that consists of a large number of services running in parallel and distributed over a cluster of computers;
%how to reduce the communication overhead caused by executing a large number of small services;
%how to support the flexible deployment of services to a network to achieve system load balance.

% citar curso alura


% Desvantagens dos microserviços:
% - Maior complexidade de desenvolvimento e infraestrutura;
% - Debug mais complexo;
% - Comunicação entre os serviços deve ser bem pensada;
% - Diversas tecnologias pode trazer problemas por inexperiência dos devs;
% - Monitoramento é crucial e mais complexo;
% - (Criar um microserviço pode ser complexo, e ter demais pode trazer problemas.).

% \subsection{Comunicação}

% - Comunicação entre os serviços deve ser bem pensada

% cross-platform compatibility issues and inconsistent call standards issues in the process of development and call microservices. \cite{ZUO2020102878}

% \subsection{[re]Organizaçao}

% Organizar o sistema e o time para sustentar uma arquitetura de microsserviços é um grande desafio. Como explica \citeonline{Familiar2015}: 
% \begin{citacao}
%     If you are part of a command-and-control organization using a waterfall software project management approach, you will struggle because you are not oriented to high-velocity product development. If you lack a DevOps culture and there is no collaboration between development and operations to automate the deployment pipeline, you will struggle. \cite{Familiar2015}
% \end{citacao}

% \subsection{Plataforma}
% Criar o ambiente de execução para microsserviços requer um grande investimento em infraestrutura dinâmica em \emph{data centers} dispersos para garantir maior disponibilidade. Se sua atual plataforma \emph{on-premises} não suporta automação, infraestrutura dinâmica, escalamento elástico e alta disponibilidade, deve-se considerar uma plataforma na nuvem. \cite{Familiar2015}. Mais sobre soluções na nuvem será discutido no \autoref{chapter-ferramentas}.

% % \subsection{Monitoramento}

% \subsection{Descoberta}

% Encontrar microsserviços em um ambiente distribuido pode ser feito de algumas maneiras diferentes. A informação pode ser armazenada diretamente no código, pode ser guardada e acessada em um arquivo, ou pode ser construido um microsserviço para encontrar outros microsserviços e disponibilizar suas localizações. Contudo, para prover detectabilidade como um serviço será necessário adquirir um produto de terceiros, integrar um projeto aberto, ou desenvolver sua própria solução. \cite{Familiar2015}
